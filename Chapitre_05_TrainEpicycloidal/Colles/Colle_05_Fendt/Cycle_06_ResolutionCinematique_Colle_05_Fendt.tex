\documentclass[10pt,fleqn]{article} % Default font size and left-justified equations
\usepackage[%
    pdftitle={CIN : Vérification des performances cinématiques des systèmes},
    pdfauthor={Xavier Pessoles}]{hyperref}
    
\input{style/new_style}
\input{style/macros_SII}

\usepackage{multicol}
\fichetrue
%\fichefalse

\proftrue
%\proffalse

\tdtrue
%\tdfalse

\courstrue
\coursfalse

\def\discipline{Sciences \\Industrielles de \\ l'Ingénieur}
\def\xxtete{Sciences Industrielles de l'Ingénieur}

\def\classe{PTSI}
\def\xxnumpartie{Cycle 8}
\def\xxpartie{Vérification des performances cinématiques des systèmes\\
Analyser, Modéliser, Résoudre}

\def\xxnumchapitre{Chapitre 5}
\def\xxchapitre{Étude des trains épicycloïdaux}

\def\xxtitreexo{Transmission à variation continue -- Vario Fendt.}
\def\xxsourceexo{\hspace{.2cm} D'après concours CCP -- MP 2008.}


\def\xxposongletx{2}
\def\xxposonglettext{1.45}
\def\xxposonglety{20}
\def\xxonglet{Cycle 6 -- Ch. 5}

\def\xxactivite{Colle 5}
\def\xxauteur{\textsl{Xavier Pessoles}}

\def\xxcompetences{%
\textsl{%
\textbf{Savoirs et compétences :}\\
\noindent \textbf{Analyser :} 
\begin{itemize}[label=\ding{112},font=\color{ocre}] 
\item \textit{A3 -- C6 :} transmetteurs de puissance.
\end{itemize}
\noindent \textbf{Modéliser :} \textit{proposer un modèle de connaissance du système.}
}}

\def\xxfigures{
\includegraphics[width=.7\textwidth]{images/fendt_01}
}%figues de la page de garde

\def\xxpied{%
Cycle 6 -- Vérification des performances cinématiques \\
Ch. 5 : Étude des trains épicycloïdaux -- \xxactivite%
}


\setcounter{secnumdepth}{5}
%---------------------------------------------------------------------------


\begin{document}
%\chapterimage{png/Fond_Cin}
\input{style/new_pagegarde}
\vspace{7cm}
\pagestyle{fancy}
\thispagestyle{plain}


\def\columnseprulecolor{\color{ocre}}
\setlength{\columnseprule}{0.4pt} 

\begin{multicols}{2}

\begin{obj} Déterminer la vitesse d'un moteur pour répondre au cahier des charges. 
\end{obj}


On s'intéresse à la chaîne de transmission de puissance d'un tracteur Fendt. Cette dernière est composée d'un moteur (et d'une pompe) hydraulique (Mh) ainsi que d'un moteur thermique MAN (Mm). 

Le moteur MAN a pour but de fournir de la puissance à la pompe hydraulique et au tracteur (récepteur R). On donne ci-dessous le schéma de la transmission. 
 
\begin{center}
\includegraphics[width=\linewidth]{images/fendt_02}
\end{center}


\begin{center}
\includegraphics[width=\linewidth]{images/fendt_03}
\end{center}


\subparagraph{}
\textit{Déterminer alors la fréquence de rotation que doit avoir le moteur <<rel>> pour respecter l'exigence 1.1.}
\end{multicols}

\end{document}


