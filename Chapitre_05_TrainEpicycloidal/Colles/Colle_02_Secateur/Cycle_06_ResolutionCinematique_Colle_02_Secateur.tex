\documentclass[10pt,fleqn]{article} % Default font size and left-justified equations
\usepackage[%
    pdftitle={CIN : Vérification des performances cinématiques des systèmes},
    pdfauthor={Xavier Pessoles}]{hyperref}
    
\input{style/new_style}
\input{style/macros_SII}

\usepackage{multicol}
\fichetrue
%\fichefalse

\proftrue
%\proffalse

\tdtrue
%\tdfalse

\courstrue
\coursfalse

\def\discipline{Sciences \\Industrielles de \\ l'Ingénieur}
\def\xxtete{Sciences Industrielles de l'Ingénieur}

\def\classe{PTSI}
\def\xxnumpartie{Cycle 8}
\def\xxpartie{Vérification des performances cinématiques des systèmes\\
Analyser, Modéliser, Résoudre}

\def\xxnumchapitre{Chapitre 5}
\def\xxchapitre{Étude des trains épicycloïdaux}

\def\xxtitreexo{Sécateur Pellenc}
\def\xxsourceexo{\hspace{.2cm} D'après ressources de Florestan Mathurin.}


\def\xxposongletx{2}
\def\xxposonglettext{1.45}
\def\xxposonglety{20}
\def\xxonglet{Cycle 6 -- Ch. 5}

\def\xxactivite{Colle 2}
\def\xxauteur{\textsl{Florestan Mathurin}}

\def\xxcompetences{%
\textsl{%
\textbf{Savoirs et compétences :}\\
\noindent \textbf{Analyser :} 
\begin{itemize}[label=\ding{112},font=\color{ocre}] 
\item \textit{A3 -- C6 :} transmetteurs de puissance.
\end{itemize}
\noindent \textbf{Modéliser :} \textit{proposer un modèle de connaissance du système.}
}}

\def\xxfigures{
\includegraphics[width=.7\textwidth]{images/secateur1}
}%figues de la page de garde

\def\xxpied{%
Cycle 6 -- Vérification des performances cinématiques \\
Ch. 5 : Étude des trains épicycloïdaux -- \xxactivite%
}


\setcounter{secnumdepth}{5}
%---------------------------------------------------------------------------


\begin{document}
%\chapterimage{png/Fond_Cin}
\input{style/new_pagegarde}
\vspace{7cm}
\pagestyle{fancy}
\thispagestyle{plain}


\def\columnseprulecolor{\color{ocre}}
\setlength{\columnseprule}{0.4pt} 

\begin{multicols}{2}

\begin{obj}~\\
Vérifier les performances d'un réducteur.
\end{obj}


La période de taille de la vigne dure 2 mois environ. Les viticulteurs coupent 9 à 10 heures par jour. Ils répètent donc le même geste des millions de fois avec un sécateur. Les sociétés réalisant le du matériel agricole ont imaginé un sécateur électrique capable de réduire la fatigue de la main et du bras tout en laissant au viticulteur la commande de la coupe et sa liberté de mouvement. Le sécateur développé par la société Pellenc permet notamment de réaliser 60 coupes de diamètre 22 mm par minute. L’ensemble sécateur Pellenc est constitué d’un sécateur électronique, d’une mallette source d’énergie, d’une sacoche avec harnais et ceinture et d’un chargeur de batterie.


Lorsque l’utilisateur appuie sur la gâchette, le moteur transmet par l’intermédiaire d’un réducteur à train épicycloïdal un mouvement de rotation à la vis à billes. L’écrou se déplace en translation par rapport à la vis et par l’intermédiaire d’une biellette met en rotation la lame mobile générant ainsi le mouvement de coupe. 

\begin{center}
 \includegraphics[width=.95\linewidth]{images/secateur2}
\end{center}




Le moteur tourne à la vitesse de rotation $N_1=1\,400\;\text{tr/min}$ le (le rotor est lié au planétaire 1). La vis à billes liée au porte-satellite 4 tourne à la vitesse de rotation $N_4=350^; \text{tr/min}$. On note $Z_1$ le nombre dents du planétaire 1, $Z_2$ celui du satellite 2 et $Z_3$ celui de la couronne liée au bâti.

\begin{center}
 \includegraphics[width=.6\linewidth]{images/secateur3}
\end{center}



\subparagraph{}
\textit{Déterminer alors le rapport de réduction du train épicycloïdal $\omega(4/0)/\omega(1/0)$ en fonction de $Z_1$ et $Z_3$.}

\subparagraph{}
\textit{Faire l’application numérique et déterminer une relation entre $Z_1$ et $Z_3$. Sachant que $Z_1=19$ en déduire $Z_3$.}

\subparagraph{}
\textit{Sachant que les roues dentées du train ont les mêmes modules, déterminer une relation géométrique entre les diamètres des éléments dentés $d_1$, $d_2$, $d_3$ puis en déduire une relation entre $Z_2$, $Z_1$, $Z_3$ (condition d’entraxe). Calculer la valeur de $Z_2$.}



\end{multicols}

\end{document}


