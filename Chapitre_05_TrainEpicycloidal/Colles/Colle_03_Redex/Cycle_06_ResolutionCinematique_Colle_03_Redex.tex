\documentclass[10pt,fleqn]{article} % Default font size and left-justified equations
\usepackage[%
    pdftitle={CIN : Vérification des performances cinématiques des systèmes},
    pdfauthor={Xavier Pessoles}]{hyperref}
    
\input{style/new_style}
\input{style/macros_SII}

\usepackage{multicol}
\fichetrue
%\fichefalse

\proftrue
%\proffalse

\tdtrue
%\tdfalse

\courstrue
\coursfalse

\def\discipline{Sciences \\Industrielles de \\ l'Ingénieur}
\def\xxtete{Sciences Industrielles de l'Ingénieur}

\def\classe{PTSI}
\def\xxnumpartie{Cycle 8}
\def\xxpartie{Vérification des performances cinématiques des systèmes\\
Analyser, Modéliser, Résoudre}

\def\xxnumchapitre{Chapitre 5}
\def\xxchapitre{Étude des trains épicycloïdaux}

\def\xxtitreexo{Poulie Redex}
\def\xxsourceexo{\hspace{.2cm} D'après ressources de Stéphane Genouël.}


\def\xxposongletx{2}
\def\xxposonglettext{1.45}
\def\xxposonglety{20}
\def\xxonglet{Cycle 6 -- Ch. 5}

\def\xxactivite{Colle 3}
\def\xxauteur{\textsl{Stéphane Genouël}}

\def\xxcompetences{%
\textsl{%
\textbf{Savoirs et compétences :}\\
\noindent \textbf{Analyser :} 
\begin{itemize}[label=\ding{112},font=\color{ocre}] 
\item \textit{A3 -- C6 :} transmetteurs de puissance.
\end{itemize}
\noindent \textbf{Modéliser :} \textit{proposer un modèle de connaissance du système.}
}}

\def\xxfigures{
\includegraphics[width=.7\textwidth]{images/fig_00}
}%figues de la page de garde

\def\xxpied{%
Cycle 6 -- Vérification des performances cinématiques \\
Ch. 5 : Étude des trains épicycloïdaux -- \xxactivite%
}


\setcounter{secnumdepth}{5}
%---------------------------------------------------------------------------


\begin{document}
%\chapterimage{png/Fond_Cin}
\input{style/new_pagegarde}
\vspace{7cm}
\pagestyle{fancy}
\thispagestyle{plain}


\def\columnseprulecolor{\color{ocre}}
\setlength{\columnseprule}{0.4pt} 

%\begin{multicols}{2}

\begin{obj}~\\
Vérifier les performances d'un réducteur.
\end{obj}

Il existe 2 grandes familles de poulies Redex H ou SR, selon la forme de l’arbre central.

%\begin{center}
%\includegraphics[width=.95\textwidth]{images/fig_01}
%\end{center}



Le mouvement d’entrée est reçu par le boîtier tournant 5, entraîné par 5 courroies trapézoïdales 8, et guidé en rotation par rapport au bâti 18 à l’aide de deux roulements à billes 23 et 28. 
Les flasques 16 permettent le montage des organes intérieurs. Ils sont munis de joints d’étanchéité 22 et 29. 
Les trois axes 9, guidés en rotation par rapport au boîtier tournant 5 à l’aide de deux roulements à aiguilles 4 et 11, portent les trois satellites doubles 6-10.
Les liaisons encastrements entre les axes 9 et les satellites 6 et 10 sont assurées (élastiquement) par de la matière plastique injectée entre les axes et les pignons préalablement dentelés (voir coupe A-A et B-B). 
Les satellites 10 sont en prise avec le planétaire 24 (qui est en liaison encastrement avec le bâti 18 à l’aide d’un assemblage cannelé).
Les satellites 6 sont en prise avec le planétaire 31 (qui est en liaison encastrement avec l’arbre de sortie 32 à l’aide d’un assemblage cannelé). Cet arbre de sortie 32 est guidé en rotation par rapport au bâti 18 à l’aide de deux roulements à aiguilles 19 et 21.



\subparagraph{}
\textit{Déterminer littéralement, en fonction des nombres de dents, la loi E/S du système (c'est-à-dire le rapport de transmission).}


%
%\begin{center}
%\includegraphics[width=.8\textwidth]{images/fig_02}
%\end{center}
%
%

\begin{center}
\includegraphics[width=.8\textwidth]{images/fig_04}
\end{center}

%\end{multicols}

\end{document}


